\documentclass{article}
\usepackage{amsmath, amsthm, amssymb}
\usepackage{listings}
\usepackage{graphicx}
\usepackage{float}
\usepackage{enumerate}
\usepackage{fancyhdr}
\usepackage{syntax}
\usepackage{synttree}
\usepackage[left=0.75in, top=1in, right=0.75in, bottom=1in]{geometry}
\pagestyle{plain}
\begin{document}
    \rhead{Jeremy Wright\\CSE 340: Homework 5}
    \thispagestyle{fancy}

    % The list environment is just to get some vertical spacing
    \list{} \item \endlist

    % Let the homework begin!
    \section*{9.6}
    \lstinputlisting{ShortCircut.cpp}

    \section*{9.14}
    Given the functions:
    \begin{lstlisting}
        cube(x)
        {
            return x*x*x;
        }

        sum(x,y,z)
        {
            return x + y + z;
        }
    \end{lstlisting}

    \subsection*{Applicative Order}
    \[sum(cube(2), cube(3), cube(4));\]
    Reduces to:  \[sum( 8, 27, 64);\] \\
    Reduces to: \[8 + 27 + 64\] \\
    6 multiplies, 2 additions.

    \subsection*{Normal Order}
    \[sum(cube(2), cube(3), cube(4));\] 
    Reduces to: \[cube(2) + cube(3) + cube(4) \]
    Reduces to: \[(2*2*2) + cube(3) + cube(4) \] 
    Reduces to: \[(8) + cube(3) + cube(4)     \]
    Reduces to: \[(8) + cube(3) + cube(4)     \]
    Reduces to: \[(8) + (3*3*3) + cube(4)     \]
    Reduces to: \[(8) + (27) + cube(4)        \]
    Reduces to: \[(8) + (27) + (4*4*4)     \]
    Reduces to: \[(8) + (27) + (64)     \]
    Reduces to: \[ 99 \]
    6 multiplies, 2 additions.


    \section*{10.5}
    Given the functions:
    \begin{lstlisting}
        void init(int a[], int size)
        {
            a = (int*) malloc(size*sizeof(int));
        }
    \end{lstlisting}
    a is not a normal pointer. a is declared as: int * const a.  As such the
    pointer cannot be reseated.

    \section*{10.9}
    \begin{lstlisting}
        int i;
        int a[3];

        void swap(int x, int y)
        {
            x = x + y;
            y = x - y;
            x = x - y;
        }

        main()
        {
            i = 1;
            a[0] = 2;
            a[1] = 1;
            a[2] = 0;

            swap(i, a[i]);
            printf("%d %d %d %d\n", i, a[0], a[1], a[2]);
            swap(a[i], a[i]);
            printf("%d %d %d\n", a[0], a[1], a[2]);
            return 0;
        }
    \end{lstlisting}
    \subsection*{Pass-By-Value}
    \begin{lstlisting}
     void swap(1, 1)
        {
            x = 1 + 1; // x <- 2
            y = 2 - 1; // y <- 1
            x = 2 - 1; // x <- 1
        }
    \end{lstlisting}
    But this implementation of swap requires references, so it does nothing.
    Output: 1 2 1 0
    Output: 2 1 0
    \subsection*{Pass-By-Reference}
    
    \begin{lstlisting}
        void swap(int& i, int& a[1])
        {
            x = x + y; // i <- x <- 1 + 1 = 2 
            y = x - y; // a[1] <- y <- 2 - 1 = 1
            x = x - y; // i <- x <- 2 - 1 = 1
            // Passes 1 back to i, and 1 back to a[1]
        }
    \end{lstlisting}
    Output: 1 2 1 0
    Output: 2 0 0
    \subsection*{Pass-By-Value-Result}
    Equivalent to: 
    \begin{lstlisting}
        void swapByValueResult(int& x, int& y)
        {
        //Setup Temporaries
        int tx = x;
        int ty = y;
        
        tx = tx + ty;
        ty = tx - ty;
        tx = tx - ty;

        //Copy the Values back
        x = tx;
        y = ty;

        }
    \end{lstlisting}
    Output: 1 2 1 0
    Output: 2 1 0
    \subsection*{Pass-By-Name}
    Equivalent to:
    \begin{lstlisting}
        #define swapByName(x, y) \
        x = x + y;\
        y = x - y;\
        x = x - y;
    \end{lstlisting}
    Output: 0 2 1 2
    Output: 0 1 2
    
    \section*{10.17}
    \begin{lstlisting}
        int x;

        void p( void )
        {
            double r = 2;
            printf(``%g\n'', r);
            printf(``%d\n'', x);
        }

        void r(void)
        {
            x = 1;
            p();
        }

        void q(void)
        {
            double x = 3;
            r ();
            p( );
        }

        main()
        {
            p( );
            q( );
            return 0;
        }
    \end{lstlisting}
    \includegraphics[width=\textwidth]{1017.png}

    \section*{10.19}
    \lstinputlisting{params.adb}
    Outputs: 2
    m +n m is passed in as 2, but n is passed as a reference through the access
    keyword. This puts the result of g on the stdout. 
    \section*{10.27}
    \subsection*{a}
    This rule could be checked statically. The activation record could be used
    to determine the location of the symbol x. If the symbol x is in the current
    activation record, then we could fail compilation.
    \subsection*{b}
    No this does not solve the dangling reference problem. Consider the
    following:
    \begin{lstlisting}
        int main()
        {
            int* x = new int;
            int* y = x;
            delete x;
            return 0;
        }
    \end{lstlisting}
    When x is delete, the symbol X is set to NULL, y however is now a dangling
    reference. This cannot be checked statically by the suggested rule.
    \end{document}
